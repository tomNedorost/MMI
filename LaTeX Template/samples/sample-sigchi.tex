%%
%% This is file `sample-sigchi.tex',
%% generated with the docstrip utility.
%%
%% The original source files were:
%%
%% samples.dtx  (with options: `sigchi')
%% 
%% IMPORTANT NOTICE:
%% 
%% For the copyright see the source file.
%% 
%% Any modified versions of this file must be renamed
%% with new filenames distinct from sample-sigchi.tex.
%% 
%% For distribution of the original source see the terms
%% for copying and modification in the file samples.dtx.
%% 
%% This generated file may be distributed as long as the
%% original source files, as listed above, are part of the
%% same distribution. (The sources need not necessarily be
%% in the same archive or directory.)
%%
%% The first command in your LaTeX source must be the \documentclass command.
\documentclass[sigchi]{acmart}
\usepackage{lineno}

%%
%% \BibTeX command to typeset BibTeX logo in the docs
\AtBeginDocument{%
  \providecommand\BibTeX{{%
    \normalfont B\kern-0.5em{\scshape i\kern-0.25em b}\kern-0.8em\TeX}}}

%% Rights management information.  This information is sent to you
%% when you complete the rights form.  These commands have SAMPLE
%% values in them; it is your responsibility as an author to replace
%% the commands and values with those provided to you when you
%% complete the rights form.
\setcopyright{acmcopyright}
\copyrightyear{2018}
\acmYear{2018}
\acmDOI{10.1145/1122445.1122456}

%% These commands are for a PROCEEDINGS abstract or paper.
\acmConference[Regensburg '19]{Regensburg '19: Social Acceptance of Nomadic Virtual
  Reality }{June 02--??, 2019}{Bayern, DE}
\acmBooktitle{Regensburg '19: Social Acceptance of Nomadic Virtual
  Reality, June 02--??, 2019, Bayern, DE}
\acmPrice{00.00}

%%
%% Submission ID.
%% Use this when submitting an article to a sponsored event. You'll
%% receive a unique submission ID from the organizers
%% of the event, and this ID should be used as the parameter to this command.
%%\acmSubmissionID{123-A56-BU3}

%%
%% The majority of ACM publications use numbered citations and
%% references.  The command \citestyle{authoryear} switches to the
%% "author year" style.
%%
%% If you are preparing content for an event
%% sponsored by ACM SIGGRAPH, you must use the "author year" style of
%% citations and references.
%% Uncommenting
%% the next command will enable that style.
%%\citestyle{acmauthoryear}

%%
%% end of the preamble, start of the body of the document source.
\begin{document}

%%
%% The "title" command has an optional parameter,
%% allowing the author to define a "short title" to be used in page headers.
\title{Social Acceptance of Nomadic Virtual Reality}

%%
%% The "author" command and its associated commands are used to define
%% the authors and their affiliations.
%% Of note is the shared affiliation of the first two authors, and the
%% "authornote" and "authornotemark" commands
%% used to denote shared contribution to the research.
\author{Alexander Eder}
\email{Alexander.Eder@stud.uni-regensburg.de}
\affiliation{%
  \institution{Universität Regensburg}
  \streetaddress{Universitätsstraße 31}
  \city{Regensburg}
  \state{Deutschland}
  \postcode{93053}
}

\author{Stephan Jäger}
\email{Stephan.Jaeger@stud.uni-regensburg.de}
\affiliation{%
  \institution{Universität Regensburg}
  \streetaddress{Universitätsstraße 31}
  \city{Regensburg}
  \state{Deutschland}
  \postcode{93053}
}

\author{Tom Nedorost}
\email{Alexander-Tom.Nedorost@stud.uni-regensburg.de}
\affiliation{%
  \institution{Universität Regensburg}
  \streetaddress{Universitätsstraße 31}
  \city{Regensburg}
  \state{Deutschland}
  \postcode{93053}
}

%%
%% By default, the full list of authors will be used in the page
%% headers. Often, this list is too long, and will overlap
%% other information printed in the page headers. This command allows
%% the author to define a more concise list
%% of authors' names for this purpose.
\renewcommand{\shortauthors}{Eder and Jäger and Nedorost}
\linenumbers

%%
%% The abstract is a short summary of the work to be presented in the
%% article.
\begin{abstract}
Lorem ipsum dolor sit amet, consetetur sadipscing elitr, sed diam nonumy eirmod tempor invidunt ut labore et dolore magna aliquyam erat, sed diam voluptua. At vero eos et accusam et justo duo dolores et ea rebum. Stet clita kasd gubergren, no sea takimata sanctus est Lorem ipsum dolor sit amet. Lorem ipsum dolor sit amet, consetetur sadipscing elitr, sed diam nonumy eirmod tempor invidunt ut labore et dolore magna aliquyam erat, sed diam voluptua. At vero eos et accusam et justo duo dolores et ea rebum. Stet clita kasd gubergren, no sea takimata sanctus est Lorem ipsum dolor sit amet. Lorem ipsum dolor sit amet, consetetur sadipscing elitr, sed diam nonumy eirmod tempor invidunt ut labore et dolore magna aliquyam erat, sed diam voluptua. At vero eos et accusam et justo duo dolores et ea rebum. Stet clita kasd gubergren, no sea takimata sanctus est Lorem ipsum dolor sit amet. 

Duis autem vel eum iriure dolor in hendrerit in vulputate velit esse molestie consequat, vel illum dolore eu feugiat nulla facilisis at vero eros et accumsan et iusto odio dignissim qui blandit praesent luptatum zzril delenit augue duis dolore te feugait nulla facilisi. Lorem ipsum dolor sit amet, consectetuer adipiscing elit, sed diam nonummy nibh euismod tincidunt ut laoreet dolore magna aliquam erat volutpat. 
\end{abstract}

%%
%% The code below is generated by the tool at http://dl.acm.org/ccs.cfm.
%% Please copy and paste the code instead of the example below.
%%
\begin{CCSXML}
<ccs2012>
 <concept>
  <concept_id>10010520.10010553.10010562</concept_id>
  <concept_desc>Computer systems organization~Embedded systems</concept_desc>
  <concept_significance>500</concept_significance>
 </concept>
 <concept>
  <concept_id>10010520.10010575.10010755</concept_id>
  <concept_desc>Computer systems organization~Redundancy</concept_desc>
  <concept_significance>300</concept_significance>
 </concept>
 <concept>
  <concept_id>10010520.10010553.10010554</concept_id>
  <concept_desc>Computer systems organization~Robotics</concept_desc>
  <concept_significance>100</concept_significance>
 </concept>
 <concept>
  <concept_id>10003033.10003083.10003095</concept_id>
  <concept_desc>Networks~Network reliability</concept_desc>
  <concept_significance>100</concept_significance>
 </concept>
</ccs2012>
\end{CCSXML}

\ccsdesc[500]{Computer systems organization~Embedded systems}
\ccsdesc[300]{Computer systems organization~Redundancy}
\ccsdesc{Computer systems organization~Robotics}
\ccsdesc[100]{Networks~Network reliability}

%%
%% Keywords. The author(s) should pick words that accurately describe
%% the work being presented. Separate the keywords with commas.
\keywords{virtual reality, social acceptance, nomadic, field study}


%%
%% This command processes the author and affiliation and title
%% information and builds the first part of the formatted document.
\maketitle

\section{Introduction}
New presentation methods like VR experience a growing trend as alternatives to conventional screens in different terminals like tablets or mobile phones. These devices are always improving in measurements, functionality, and appearance because of this, to accommodate the mobility of modern life. Although the development process of them is still far away from being finished VR devices might be prospectively used in the same way we already use mobile phones today, at any time and everywhere. To achieve a broad utilization, it is not only important to focus on the unique user and establish hardware with high usability for the users themselves, but also something that fits all the tangentially involved people and their needs for well-being, comfort, and privacy. The most important issue to start with, which also is the topic of this paper, is the question about the current state of social acceptance of VR devices in public spaces. Before spreading out this type of gear and gaining the possibility of high sales output it is essential to find out if those devices are already accepted by society and which impacts they have on society.

In the paper “Virtual reality on the go?: a study on the social acceptance of VR glasses” \cite{schwind2018virtual} several researchers already tried to investigate this potential issue by showing pictures and videos of people wearing VR devices in public spaces to a group of test persons under laboratory conditions to find out more about their opinions, feelings, and reactions confronted with this subject. As we all know it is hard to put oneself in a position you only see on pictures. With the inspection of images, people will always keep a certain emotional distance to the context shown. 
The spontaneous confrontation with a previously completely unexpected situation in daily life might have another effect on their emotional acceptance. VR devices might be fully accepted by society, but it can also be that they evoke discomfort because people are not used to not see each other's eyes while passing by or sitting next to them on the bench. Sunglasses of course act similar but since today´s VR goggles still, cover almost half of the wearers face it cannot be generalized and needs to be examined more accurate.
In this paper, the mentioned issue will be reexamined using a field study to achieve a high validity not only in the laboratory but also in the open field.

\section{Related Work}
As noted in the introduction, the ``\verb|acmart|'' document class can
be used to prepare many different kinds of documentation --- a
double-blind initial submission of a full-length technical paper, a
two-page SIGGRAPH Emerging Technologies abstract, a ``camera-ready''
journal article, a SIGCHI Extended Abstract, and more --- all by
selecting the appropriate {\itshape template style} and {\itshape
  template parameters}.

This document will explain the major features of the document
class. For further information, the {\itshape \LaTeX\ User's Guide} is
available from
\url{https://www.acm.org/publications/proceedings-template}.

\section{STUDY: Acceptance of Nomadic Virtual Reality}

As already mentioned VR devices represent a potential upcoming alternative to conventional screens in the mobile context. The specific target of this study was to examine more about the current state of social acceptance in the open field by confronting unprepared pedestrians with this topic in different real life scenarios. This was done with the help of a field study because of our hypothesis that the procedure under laboratory conditions will have another result due to emotional distances.

\subsection{Study Design}

The design of the study was a two-factorial within-subject design and conducted with the help of the three independent variables GENDER, WEARING OF VR-GOGGLES and PERFORMING GESTURES. The usage of VR devices does not only include the actual wearing of the goggles. Gesture control with the help of connected VR controllers is essential for the use of VR devices of any kind. Since performing those gestures might have a big impact on the acceptance this also was a very important issue to test to find out more about the general acceptance and how people react when beeing contfronted with this situation. It is also important to investigate whether the gender of the wearer has an influence on the results or not.

\subsection{Stimuli}

In earlier researches pictures and videoclips have been used for probing \cite{schwind2018virtual}. Since we wanted to extend those results and test their external validity we used confrontations in real life situations in the open field rather than representations of it. The first important stimuli was the gender. We wanted to find out if the gender itself plays an important role with the acceptance of such devices in general. Both genders have been tested without using any VR tools to get a baseline for upcoming steps and procedures. Another stimuli we used was the fact that both our actor and our actress will wear a VR goggle to test its influence on the pedestrians. Last but not least we tested the goggles in combination with controllers and gesture controls which is our final stimuli. In this study we combined those three stimuli to receive as much information as possible about peoples reactions on different types of situations.

\subsection{Survey Procedure}

After handing out the informed consent, the randomly chosen participants obtained a short demographic questionaire in which we request allegations to gender, age and job. Afterwards we handed out another Questionaire to measure the acceptability of wearable devices \cite{kelly2016wear}. The socalled WEAR Scale is a questionaire that consists of several items to finde out how acceptable a device is with regard to e.g. asthetic itself, the wearers charisma it awakes and the own attitude towards the gadget. Subsequent each participant received a little thank-you gift.

\subsection{Participants}

Because this study has not been researched under laboratory conditions it was not possible to recruit test persons. Another reason for us to not hire subjects was, that this would have not lead to the result we were looking for. We wanted to examine this Acceptance Rating by collection real life reactions and the opinions they might bear on. For this type of field study it was essential to blindside pedestrians in their daily life to receive an unbiased output. 

%%
%% The next two lines define the bibliography style to be used, and
%% the bibliography file.
\bibliographystyle{ACM-Reference-Format}
\bibliography{sample-base}

%%
%% If your work has an appendix, this is the place to put it.
\appendix

\end{document}
\endinput
%%
%% End of file `sample-sigchi.tex'.
